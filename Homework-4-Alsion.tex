% Options for packages loaded elsewhere
\PassOptionsToPackage{unicode}{hyperref}
\PassOptionsToPackage{hyphens}{url}
%
\documentclass[
]{article}
\usepackage{lmodern}
\usepackage{amssymb,amsmath}
\usepackage{ifxetex,ifluatex}
\ifnum 0\ifxetex 1\fi\ifluatex 1\fi=0 % if pdftex
  \usepackage[T1]{fontenc}
  \usepackage[utf8]{inputenc}
  \usepackage{textcomp} % provide euro and other symbols
\else % if luatex or xetex
  \usepackage{unicode-math}
  \defaultfontfeatures{Scale=MatchLowercase}
  \defaultfontfeatures[\rmfamily]{Ligatures=TeX,Scale=1}
\fi
% Use upquote if available, for straight quotes in verbatim environments
\IfFileExists{upquote.sty}{\usepackage{upquote}}{}
\IfFileExists{microtype.sty}{% use microtype if available
  \usepackage[]{microtype}
  \UseMicrotypeSet[protrusion]{basicmath} % disable protrusion for tt fonts
}{}
\makeatletter
\@ifundefined{KOMAClassName}{% if non-KOMA class
  \IfFileExists{parskip.sty}{%
    \usepackage{parskip}
  }{% else
    \setlength{\parindent}{0pt}
    \setlength{\parskip}{6pt plus 2pt minus 1pt}}
}{% if KOMA class
  \KOMAoptions{parskip=half}}
\makeatother
\usepackage{xcolor}
\IfFileExists{xurl.sty}{\usepackage{xurl}}{} % add URL line breaks if available
\IfFileExists{bookmark.sty}{\usepackage{bookmark}}{\usepackage{hyperref}}
\hypersetup{
  pdftitle={ESM 204 HW 4: Calculating the SCC and policy choice under uncertainty},
  pdfauthor={Larissa Neilson, Alison Sells, Katelyn Toigo},
  hidelinks,
  pdfcreator={LaTeX via pandoc}}
\urlstyle{same} % disable monospaced font for URLs
\usepackage[margin=1in]{geometry}
\usepackage{color}
\usepackage{fancyvrb}
\newcommand{\VerbBar}{|}
\newcommand{\VERB}{\Verb[commandchars=\\\{\}]}
\DefineVerbatimEnvironment{Highlighting}{Verbatim}{commandchars=\\\{\}}
% Add ',fontsize=\small' for more characters per line
\usepackage{framed}
\definecolor{shadecolor}{RGB}{248,248,248}
\newenvironment{Shaded}{\begin{snugshade}}{\end{snugshade}}
\newcommand{\AlertTok}[1]{\textcolor[rgb]{0.94,0.16,0.16}{#1}}
\newcommand{\AnnotationTok}[1]{\textcolor[rgb]{0.56,0.35,0.01}{\textbf{\textit{#1}}}}
\newcommand{\AttributeTok}[1]{\textcolor[rgb]{0.77,0.63,0.00}{#1}}
\newcommand{\BaseNTok}[1]{\textcolor[rgb]{0.00,0.00,0.81}{#1}}
\newcommand{\BuiltInTok}[1]{#1}
\newcommand{\CharTok}[1]{\textcolor[rgb]{0.31,0.60,0.02}{#1}}
\newcommand{\CommentTok}[1]{\textcolor[rgb]{0.56,0.35,0.01}{\textit{#1}}}
\newcommand{\CommentVarTok}[1]{\textcolor[rgb]{0.56,0.35,0.01}{\textbf{\textit{#1}}}}
\newcommand{\ConstantTok}[1]{\textcolor[rgb]{0.00,0.00,0.00}{#1}}
\newcommand{\ControlFlowTok}[1]{\textcolor[rgb]{0.13,0.29,0.53}{\textbf{#1}}}
\newcommand{\DataTypeTok}[1]{\textcolor[rgb]{0.13,0.29,0.53}{#1}}
\newcommand{\DecValTok}[1]{\textcolor[rgb]{0.00,0.00,0.81}{#1}}
\newcommand{\DocumentationTok}[1]{\textcolor[rgb]{0.56,0.35,0.01}{\textbf{\textit{#1}}}}
\newcommand{\ErrorTok}[1]{\textcolor[rgb]{0.64,0.00,0.00}{\textbf{#1}}}
\newcommand{\ExtensionTok}[1]{#1}
\newcommand{\FloatTok}[1]{\textcolor[rgb]{0.00,0.00,0.81}{#1}}
\newcommand{\FunctionTok}[1]{\textcolor[rgb]{0.00,0.00,0.00}{#1}}
\newcommand{\ImportTok}[1]{#1}
\newcommand{\InformationTok}[1]{\textcolor[rgb]{0.56,0.35,0.01}{\textbf{\textit{#1}}}}
\newcommand{\KeywordTok}[1]{\textcolor[rgb]{0.13,0.29,0.53}{\textbf{#1}}}
\newcommand{\NormalTok}[1]{#1}
\newcommand{\OperatorTok}[1]{\textcolor[rgb]{0.81,0.36,0.00}{\textbf{#1}}}
\newcommand{\OtherTok}[1]{\textcolor[rgb]{0.56,0.35,0.01}{#1}}
\newcommand{\PreprocessorTok}[1]{\textcolor[rgb]{0.56,0.35,0.01}{\textit{#1}}}
\newcommand{\RegionMarkerTok}[1]{#1}
\newcommand{\SpecialCharTok}[1]{\textcolor[rgb]{0.00,0.00,0.00}{#1}}
\newcommand{\SpecialStringTok}[1]{\textcolor[rgb]{0.31,0.60,0.02}{#1}}
\newcommand{\StringTok}[1]{\textcolor[rgb]{0.31,0.60,0.02}{#1}}
\newcommand{\VariableTok}[1]{\textcolor[rgb]{0.00,0.00,0.00}{#1}}
\newcommand{\VerbatimStringTok}[1]{\textcolor[rgb]{0.31,0.60,0.02}{#1}}
\newcommand{\WarningTok}[1]{\textcolor[rgb]{0.56,0.35,0.01}{\textbf{\textit{#1}}}}
\usepackage{graphicx,grffile}
\makeatletter
\def\maxwidth{\ifdim\Gin@nat@width>\linewidth\linewidth\else\Gin@nat@width\fi}
\def\maxheight{\ifdim\Gin@nat@height>\textheight\textheight\else\Gin@nat@height\fi}
\makeatother
% Scale images if necessary, so that they will not overflow the page
% margins by default, and it is still possible to overwrite the defaults
% using explicit options in \includegraphics[width, height, ...]{}
\setkeys{Gin}{width=\maxwidth,height=\maxheight,keepaspectratio}
% Set default figure placement to htbp
\makeatletter
\def\fps@figure{htbp}
\makeatother
\setlength{\emergencystretch}{3em} % prevent overfull lines
\providecommand{\tightlist}{%
  \setlength{\itemsep}{0pt}\setlength{\parskip}{0pt}}
\setcounter{secnumdepth}{-\maxdimen} % remove section numbering

\title{ESM 204 HW 4: Calculating the SCC and policy choice under uncertainty}
\author{Larissa Neilson, Alison Sells, Katelyn Toigo}
\date{5/28/2021}

\begin{document}
\maketitle

\begin{Shaded}
\begin{Highlighting}[]
\CommentTok{# Read in the data}
\NormalTok{damage <-}\StringTok{ }\KeywordTok{read_csv}\NormalTok{(}\KeywordTok{here}\NormalTok{(}\StringTok{"data"}\NormalTok{, }\StringTok{"damages.csv"}\NormalTok{)) }\OperatorTok\StringTok{ }
\StringTok{  }\KeywordTok{clean_names}\NormalTok{()}
\NormalTok{warming <-}\StringTok{ }\KeywordTok{read_csv}\NormalTok{(}\KeywordTok{here}\NormalTok{(}\StringTok{"data"}\NormalTok{, }\StringTok{"warming.csv"}\NormalTok{)) }\OperatorTok\StringTok{ }
\StringTok{  }\KeywordTok{clean_names}\NormalTok{()}
\end{Highlighting}
\end{Shaded}

\begin{enumerate}
\def\labelenumi{\arabic{enumi}.}
\tightlist
\item
  Using damages.csv, estimate a quadratic damage function relating the
  dollar value of damages to the change in global mean temperature. Omit
  an intercept term; damages by construction must equal zero when there
  is no climate change. Plot your estimated damage function, overlaid
  with a scatterplot of the underlying data.
\end{enumerate}

\begin{Shaded}
\begin{Highlighting}[]
\CommentTok{# Question 1}
\NormalTok{damages <-}\StringTok{ }\NormalTok{damage }\OperatorTok
\StringTok{  }\KeywordTok{mutate}\NormalTok{(}\DataTypeTok{warming2 =}\NormalTok{ warming}\OperatorTok{^}\DecValTok{2}\NormalTok{)}

\CommentTok{# Quadratic model of damages}
\NormalTok{dam_lm <-}\StringTok{ }\KeywordTok{lm}\NormalTok{(damages }\OperatorTok{~}\StringTok{ }\NormalTok{warming }\OperatorTok{+}\StringTok{ }\NormalTok{warming2, }\DataTypeTok{data =}\NormalTok{ damages)}
\NormalTok{dam_lm[[}\StringTok{"coefficients"}\NormalTok{]][[}\StringTok{"(Intercept)"}\NormalTok{]] <-}\StringTok{ }\DecValTok{0}
\NormalTok{dam_lm}
\end{Highlighting}
\end{Shaded}

\begin{verbatim}
## 
## Call:
## lm(formula = damages ~ warming + warming2, data = damages)
## 
## Coefficients:
## (Intercept)      warming     warming2  
##   0.000e+00   -3.019e+12    1.959e+13
\end{verbatim}

\begin{Shaded}
\begin{Highlighting}[]
\CommentTok{# Damages Plot}
\NormalTok{damages_plot <-}\StringTok{ }\KeywordTok{ggplot}\NormalTok{(}\DataTypeTok{data =}\NormalTok{ damages) }\OperatorTok{+}
\StringTok{  }\KeywordTok{geom_point}\NormalTok{(}\KeywordTok{aes}\NormalTok{(}\DataTypeTok{x =}\NormalTok{ warming, }\DataTypeTok{y =}\NormalTok{ damages)) }\OperatorTok{+}
\StringTok{  }\KeywordTok{stat_smooth}\NormalTok{(}\DataTypeTok{data =}\NormalTok{ damages, }\KeywordTok{aes}\NormalTok{(}\DataTypeTok{x =}\NormalTok{ warming, }\DataTypeTok{y =}\NormalTok{ damages)) }\OperatorTok{+}
\StringTok{  }\KeywordTok{labs}\NormalTok{(}\DataTypeTok{x =} \StringTok{"Level of warming (ºC)"}\NormalTok{, }\DataTypeTok{y =} \StringTok{"Annual total damages ($)"}\NormalTok{) }\OperatorTok{+}
\StringTok{  }\KeywordTok{theme_minimal}\NormalTok{()}
\CommentTok{#damages_plot}
\end{Highlighting}
\end{Shaded}

\begin{Shaded}
\begin{Highlighting}[]
\CommentTok{# Function with our quadratic equation}
\NormalTok{damage_function <-}\StringTok{ }\ControlFlowTok{function}\NormalTok{(warming) \{}
\NormalTok{  damages <-}\StringTok{ }\NormalTok{(}\DecValTok{19590000000000} \OperatorTok{*}\StringTok{ }\NormalTok{warming}\OperatorTok{^}\DecValTok{2}\NormalTok{) }\OperatorTok{-}\StringTok{ }\NormalTok{(}\DecValTok{3019000000000} \OperatorTok{*}\StringTok{ }\NormalTok{warming)}
  \KeywordTok{return}\NormalTok{(damages)}
\NormalTok{\}}

\CommentTok{# Test}
\CommentTok{#damage_function(0.3022845)}
\end{Highlighting}
\end{Shaded}

\begin{enumerate}
\def\labelenumi{\arabic{enumi}.}
\setcounter{enumi}{1}
\tightlist
\item
  Use warming.csv and your estimated damage function to predict damages
  in each year under the baseline climate and the pulse scenario. Make
  four plots: (1) damages over time without the pulse, (2) damages over
  time with the pulse, (3) the difference in damages over time that
  arises from the pulse, and (4) the difference in damages over time
  from the pulse per ton of CO2 (you can assume that each ton of the
  pulse causes the same amount of damage).
\end{enumerate}

\begin{Shaded}
\begin{Highlighting}[]
\CommentTok{# Test, this should input the same # as the last test line but it's slightly off?}
\CommentTok{#damage_function(warming[1, "warming_baseline"])}

\CommentTok{# This makes new column with the calculated damages using values from the warming_baseline column}
\NormalTok{warming}\OperatorTok{$}\NormalTok{damage_base <-}\StringTok{ }\KeywordTok{damage_function}\NormalTok{(warming}\OperatorTok{$}\NormalTok{warming_baseline)}
\CommentTok{#head(warming$damage_base)}

\CommentTok{# Question 2 plot (1)}
\NormalTok{plot_baseline <-}\StringTok{ }\KeywordTok{ggplot}\NormalTok{(}\DataTypeTok{data =}\NormalTok{ warming, }\KeywordTok{aes}\NormalTok{(}\DataTypeTok{x =}\NormalTok{ year, }\DataTypeTok{y =}\NormalTok{ damage_base)) }\OperatorTok{+}
\StringTok{  }\KeywordTok{geom_point}\NormalTok{(}\DataTypeTok{cex =} \FloatTok{0.5}\NormalTok{) }\OperatorTok{+}
\StringTok{  }\KeywordTok{labs}\NormalTok{(}\DataTypeTok{x =} \StringTok{"Year"}\NormalTok{, }\DataTypeTok{y =} \StringTok{"Total damages ($)"}\NormalTok{, }\DataTypeTok{title =} \StringTok{"Damages without Pulse"}\NormalTok{) }\OperatorTok{+}
\StringTok{  }\KeywordTok{theme_minimal}\NormalTok{()}

\CommentTok{# Do the same with values from the warming_pulse column}
\NormalTok{warming}\OperatorTok{$}\NormalTok{damage_pulse <-}\StringTok{ }\KeywordTok{damage_function}\NormalTok{(warming}\OperatorTok{$}\NormalTok{warming_pulse)}
\CommentTok{#head(warming$damage_pulse)}

\CommentTok{# Question 2 plot (2)}
\NormalTok{plot_pulse <-}\StringTok{ }\KeywordTok{ggplot}\NormalTok{(}\DataTypeTok{data =}\NormalTok{ warming, }\KeywordTok{aes}\NormalTok{(}\DataTypeTok{x =}\NormalTok{ year, }\DataTypeTok{y =}\NormalTok{ damage_pulse)) }\OperatorTok{+}
\StringTok{  }\KeywordTok{geom_point}\NormalTok{(}\DataTypeTok{cex =} \FloatTok{0.5}\NormalTok{) }\OperatorTok{+}
\StringTok{  }\KeywordTok{labs}\NormalTok{(}\DataTypeTok{x =} \StringTok{"Year"}\NormalTok{, }\DataTypeTok{y =} \StringTok{"Total damages ($)"}\NormalTok{, }\DataTypeTok{title =} \StringTok{"Damages with Pulse"}\NormalTok{) }\OperatorTok{+}
\StringTok{  }\KeywordTok{theme_minimal}\NormalTok{()}
\end{Highlighting}
\end{Shaded}

\begin{Shaded}
\begin{Highlighting}[]
\NormalTok{warming_diff <-}\StringTok{ }\NormalTok{warming }\OperatorTok\StringTok{ }
\StringTok{  }\KeywordTok{mutate}\NormalTok{(}\DataTypeTok{damage_diff =}\NormalTok{ (damage_pulse }\OperatorTok{-}\StringTok{ }\NormalTok{damage_base))}

\CommentTok{# Question 2 plot (3)}
\NormalTok{plot_difference <-}\StringTok{ }\KeywordTok{ggplot}\NormalTok{(}\DataTypeTok{data =}\NormalTok{ warming_diff, }\KeywordTok{aes}\NormalTok{(}\DataTypeTok{x =}\NormalTok{ year, }\DataTypeTok{y =}\NormalTok{ damage_diff)) }\OperatorTok{+}
\StringTok{  }\KeywordTok{geom_point}\NormalTok{(}\DataTypeTok{cex =} \FloatTok{0.5}\NormalTok{) }\OperatorTok{+}
\StringTok{  }\KeywordTok{labs}\NormalTok{(}\DataTypeTok{x =} \StringTok{"Year"}\NormalTok{, }\DataTypeTok{y =} \StringTok{"Difference in damage ($)"}\NormalTok{, }\DataTypeTok{title =} \StringTok{"Diff. in Damages with Pulse"}\NormalTok{) }\OperatorTok{+}
\StringTok{  }\KeywordTok{theme_minimal}\NormalTok{()}
\end{Highlighting}
\end{Shaded}

\begin{Shaded}
\begin{Highlighting}[]
\CommentTok{# Question 2 Part 4}
\NormalTok{warming_diff_norm<-}\StringTok{ }\NormalTok{warming_diff }\OperatorTok\StringTok{ }
\StringTok{  }\KeywordTok{mutate}\NormalTok{(}\DataTypeTok{damage_diff_norm =}\NormalTok{ damage_diff}\OperatorTok{/}\StringTok{ }\DecValTok{35000000000}\NormalTok{)}

\CommentTok{# Question 2 plot (4)}
\NormalTok{plot_difference_per_ton <-}\StringTok{ }\KeywordTok{ggplot}\NormalTok{(}\DataTypeTok{data =}\NormalTok{ warming_diff_norm, }\KeywordTok{aes}\NormalTok{(}\DataTypeTok{x =}\NormalTok{ year, }\DataTypeTok{y =}\NormalTok{ damage_diff_norm)) }\OperatorTok{+}
\StringTok{  }\KeywordTok{geom_point}\NormalTok{(}\DataTypeTok{cex =} \FloatTok{0.5}\NormalTok{) }\OperatorTok{+}
\StringTok{  }\KeywordTok{labs}\NormalTok{(}\DataTypeTok{x =} \StringTok{"Year"}\NormalTok{, }\DataTypeTok{y =} \StringTok{" Difference in damage ($ per ton)"}\NormalTok{, }\DataTypeTok{title =} \StringTok{"Diff. in Dmg. with Pulse/ton of CO2"}\NormalTok{) }\OperatorTok{+}
\StringTok{  }\KeywordTok{theme_minimal}\NormalTok{()}
\end{Highlighting}
\end{Shaded}

\begin{Shaded}
\begin{Highlighting}[]
\CommentTok{# Use patchwork package}
\NormalTok{all_plots <-}\StringTok{ }\NormalTok{(plot_baseline }\OperatorTok{+}\StringTok{ }\NormalTok{plot_pulse) }\OperatorTok{/}\StringTok{ }\NormalTok{(plot_difference }\OperatorTok{+}\StringTok{ }\NormalTok{plot_difference_per_ton)}

\NormalTok{all_plots}
\end{Highlighting}
\end{Shaded}

\includegraphics{Homework-4-Alsion_files/figure-latex/unnamed-chunk-7-1.pdf}

\begin{enumerate}
\def\labelenumi{\arabic{enumi}.}
\setcounter{enumi}{2}
\tightlist
\item
  The SCC is the present discounted value of the stream of future
  damages caused by one additional ton of CO2. The Obama Administration
  used a discount rate of 3\% to discount damages. Recently, New York
  State used a discount rate of 2\%. Calculate and make a plot of the
  SCC (y-axis) against the discount rate (x-axis) for a reasonable range
  of discount rates.
\end{enumerate}

\begin{Shaded}
\begin{Highlighting}[]
\CommentTok{# Question 3}
\NormalTok{scc <-}\StringTok{ }\NormalTok{warming_diff_norm }\OperatorTok\StringTok{ }
\StringTok{  }\KeywordTok{mutate}\NormalTok{(}\DataTypeTok{scc_1 =}\NormalTok{ damage_diff}\OperatorTok{/}\StringTok{ }\NormalTok{(}\DecValTok{1}\FloatTok{+0.01}\NormalTok{)}\OperatorTok{^}\NormalTok{x1,}
         \DataTypeTok{scc_2 =}\NormalTok{ damage_diff}\OperatorTok{/}\StringTok{ }\NormalTok{(}\DecValTok{1}\FloatTok{+0.02}\NormalTok{)}\OperatorTok{^}\NormalTok{x1,}
         \DataTypeTok{scc_3 =}\NormalTok{ damage_diff}\OperatorTok{/}\StringTok{ }\NormalTok{(}\DecValTok{1}\FloatTok{+0.03}\NormalTok{)}\OperatorTok{^}\NormalTok{x1,}
         \DataTypeTok{scc_4 =}\NormalTok{ damage_diff}\OperatorTok{/}\StringTok{ }\NormalTok{(}\DecValTok{1}\FloatTok{+0.04}\NormalTok{)}\OperatorTok{^}\NormalTok{x1,}
         \DataTypeTok{scc_5 =}\NormalTok{ damage_diff}\OperatorTok{/}\StringTok{ }\NormalTok{(}\DecValTok{1}\FloatTok{+0.05}\NormalTok{)}\OperatorTok{^}\NormalTok{x1,}
         \DataTypeTok{scc1_sum =} \KeywordTok{sum}\NormalTok{(scc_}\DecValTok{1}\NormalTok{),}
         \DataTypeTok{scc2_sum =} \KeywordTok{sum}\NormalTok{(scc_}\DecValTok{2}\NormalTok{),}
         \DataTypeTok{scc3_sum =} \KeywordTok{sum}\NormalTok{(scc_}\DecValTok{3}\NormalTok{),}
         \DataTypeTok{scc4_sum =} \KeywordTok{sum}\NormalTok{(scc_}\DecValTok{4}\NormalTok{),}
         \DataTypeTok{scc5_sum =} \KeywordTok{sum}\NormalTok{(scc_}\DecValTok{5}\NormalTok{)) }

\NormalTok{scc_df <-}\StringTok{ }\KeywordTok{data.frame}\NormalTok{(}\KeywordTok{c}\NormalTok{(}\FloatTok{0.01}\NormalTok{, }\FloatTok{0.02}\NormalTok{, }\FloatTok{0.03}\NormalTok{, }\FloatTok{0.04}\NormalTok{, }\FloatTok{0.05}\NormalTok{)) }\OperatorTok
\StringTok{  }\KeywordTok{cbind}\NormalTok{(}\KeywordTok{c}\NormalTok{(scc}\OperatorTok{$}\NormalTok{scc1_sum[}\DecValTok{1}\NormalTok{], scc}\OperatorTok{$}\NormalTok{scc2_sum[}\DecValTok{1}\NormalTok{], scc}\OperatorTok{$}\NormalTok{scc3_sum[}\DecValTok{1}\NormalTok{], scc}\OperatorTok{$}\NormalTok{scc4_sum[}\DecValTok{1}\NormalTok{], scc}\OperatorTok{$}\NormalTok{scc5_sum[}\DecValTok{1}\NormalTok{]))}
\KeywordTok{colnames}\NormalTok{(scc_df) <-}\StringTok{ }\KeywordTok{c}\NormalTok{(}\StringTok{"rate"}\NormalTok{, }\StringTok{"scc_sum"}\NormalTok{)}
\end{Highlighting}
\end{Shaded}

\begin{Shaded}
\begin{Highlighting}[]
\KeywordTok{ggplot}\NormalTok{(}\DataTypeTok{data =}\NormalTok{ scc_df) }\OperatorTok{+}
\StringTok{  }\KeywordTok{geom_point}\NormalTok{(}\KeywordTok{aes}\NormalTok{(}\DataTypeTok{x =}\NormalTok{ rate, }\DataTypeTok{y =}\NormalTok{ scc_sum)) }\OperatorTok{+}
\StringTok{  }\KeywordTok{labs}\NormalTok{(}\DataTypeTok{x =} \StringTok{"Discount rate"}\NormalTok{, }\DataTypeTok{y =} \StringTok{"Total SCC through 2100"}\NormalTok{, }\DataTypeTok{title =} \StringTok{"Impact of discount rates on the SCC"}\NormalTok{) }\OperatorTok{+}
\StringTok{  }\KeywordTok{theme_minimal}\NormalTok{()}
\end{Highlighting}
\end{Shaded}

\includegraphics{Homework-4-Alsion_files/figure-latex/unnamed-chunk-9-1.pdf}

\begin{enumerate}
\def\labelenumi{\arabic{enumi}.}
\setcounter{enumi}{3}
\tightlist
\item
  The National Academies of Sciences, Engineering, and Medicine advised
  the government in a 2017 report to use the Ramsey Rule when
  discounting within the SCC calculation: r = ρ + ηg Using ρ = 0.001, η
  = 2, and g = 0.01, what is the SCC? Locate this point on your graph
  from above.
\end{enumerate}

\begin{Shaded}
\begin{Highlighting}[]
\CommentTok{# Question 4}
\NormalTok{rams =}\StringTok{ }\FloatTok{0.001} \OperatorTok{+}\StringTok{ }\NormalTok{(}\DecValTok{2}\OperatorTok{*}\FloatTok{0.01}\NormalTok{) }\CommentTok{# new r value}

\NormalTok{scc<-}\StringTok{ }\NormalTok{warming_diff_norm }\OperatorTok\StringTok{ }
\StringTok{  }\KeywordTok{mutate}\NormalTok{(}\DataTypeTok{scc_rams =}\NormalTok{ damage_diff}\OperatorTok{/}\StringTok{ }\NormalTok{(}\DecValTok{1} \OperatorTok{+}\StringTok{ }\NormalTok{rams)}\OperatorTok{^}\NormalTok{x1,}
          \DataTypeTok{scc_rams_sum =} \KeywordTok{sum}\NormalTok{(scc_rams))}
         
\NormalTok{scc_rams_df <-}\StringTok{ }\KeywordTok{data.frame}\NormalTok{(}\KeywordTok{c}\NormalTok{(}\FloatTok{0.021}\NormalTok{)) }\OperatorTok
\StringTok{  }\KeywordTok{cbind}\NormalTok{(}\KeywordTok{c}\NormalTok{(scc}\OperatorTok{$}\NormalTok{scc_rams_sum[}\DecValTok{1}\NormalTok{]))}
\KeywordTok{colnames}\NormalTok{(scc_rams_df) <-}\StringTok{ }\KeywordTok{c}\NormalTok{(}\StringTok{"rate"}\NormalTok{, }\StringTok{"scc_sum"}\NormalTok{)}
\end{Highlighting}
\end{Shaded}

\begin{Shaded}
\begin{Highlighting}[]
\CommentTok{# This line fixes the legend to be what we want the descriptor to say}
\NormalTok{cols <-}\StringTok{ }\KeywordTok{c}\NormalTok{(}\StringTok{"New SCC"}\NormalTok{ =}\StringTok{ "tomato2"}\NormalTok{)}

\KeywordTok{ggplot}\NormalTok{() }\OperatorTok{+}
\StringTok{  }\KeywordTok{geom_point}\NormalTok{(}\DataTypeTok{data =}\NormalTok{ scc_df,}
             \KeywordTok{aes}\NormalTok{(}\DataTypeTok{x =}\NormalTok{ rate, }\DataTypeTok{y =}\NormalTok{ scc_sum)) }\OperatorTok{+}
\StringTok{  }\KeywordTok{geom_point}\NormalTok{(}\DataTypeTok{data =}\NormalTok{ scc_rams_df, }\KeywordTok{aes}\NormalTok{(}\DataTypeTok{x =}\NormalTok{ rate, }\DataTypeTok{y =}\NormalTok{ scc_sum, }\DataTypeTok{color =} \StringTok{"SCC Given the Conditions"}\NormalTok{)) }\OperatorTok{+}
\StringTok{  }\KeywordTok{theme}\NormalTok{(}\DataTypeTok{legend.title =} \KeywordTok{element_blank}\NormalTok{()) }\OperatorTok{+}
\StringTok{  }\KeywordTok{labs}\NormalTok{(}\DataTypeTok{x =} \StringTok{"Discount rate"}\NormalTok{, }\DataTypeTok{y =} \StringTok{"Total SCC through 2100"}\NormalTok{, }\DataTypeTok{title =} \StringTok{"New SCC Given the Conditions"}\NormalTok{) }\OperatorTok{+}
\StringTok{  }\KeywordTok{theme_minimal}\NormalTok{()}
\end{Highlighting}
\end{Shaded}

\includegraphics{Homework-4-Alsion_files/figure-latex/unnamed-chunk-11-1.pdf}

\begin{enumerate}
\def\labelenumi{\arabic{enumi}.}
\setcounter{enumi}{4}
\tightlist
\item
  Now suppose there are two possible climate policies that can be
  pursued. Policy A is business as usual and Policy B is to take
  immediate and strong action on climate change. Use these facts:
\end{enumerate}

\begin{itemize}
\tightlist
\item
  If you undertake Policy A there are two possible outcomes. Either
  warming will occur as in the ``baseline'' (i.e.~``no-pulse'') dataset
  above (this happens with probability 0.5) or warming each year will be
  1.5 times that in the ``baseline'' dataset (with probability 0.5).
\item
  Under Policy B, warming will continue until 2050 as in the
  ``baseline'' dataset, and then will stabilize at 1.29 degrees and stay
  that way forever.
\item
  Society is risk neutral
\item
  Use a discount rate of 2\%
\end{itemize}

\begin{enumerate}
\def\labelenumi{\Alph{enumi})}
\tightlist
\item
  What is the expected present value of damages up to 2100 under Policy
  A? B)What is the expected present value of damages up to 2100 under
  Policy B? C)Suppose undertaking Policy D)A costs zero and undertaking
  Policy B costs X. How large could X be for it to still make economic
  sense to pursue Policy B instead of Policy A? E)Qualitatively, how
  would your answer change if society were risk averse?
\end{enumerate}

\begin{Shaded}
\begin{Highlighting}[]
\CommentTok{# Question 5}
\CommentTok{# Since There are two policy outcomes for A, we need to figure out the expected present value for each and multiply each potential outcome by their respective probabilities }
\NormalTok{warming_sub <-}\StringTok{ }\NormalTok{warming }\OperatorTok\StringTok{ }
\StringTok{  }\KeywordTok{select}\NormalTok{(x1,year, warming_baseline, damage_base) }

\NormalTok{prob_dam <-}\StringTok{ }\NormalTok{warming_sub }\OperatorTok\StringTok{ }
\StringTok{  }\KeywordTok{mutate}\NormalTok{(}\DataTypeTok{warming_1.5 =}\NormalTok{ warming_baseline }\OperatorTok{*}\StringTok{ }\FloatTok{1.5}\NormalTok{)}

\NormalTok{prob_dam}\OperatorTok{$}\NormalTok{damage_}\FloatTok{1.5}\NormalTok{ <-}\StringTok{ }\KeywordTok{damage_function}\NormalTok{(prob_dam}\OperatorTok{$}\NormalTok{warming_}\FloatTok{1.5}\NormalTok{)}
\CommentTok{#head(prob_dam$damage_1.5)}

\NormalTok{sum_prob_dam <-}\StringTok{ }\NormalTok{prob_dam }\OperatorTok\StringTok{ }
\StringTok{  }\KeywordTok{mutate}\NormalTok{(}\DataTypeTok{npv_1.5 =}\NormalTok{ damage_}\FloatTok{1.5}\OperatorTok{/}\StringTok{ }\NormalTok{(}\DecValTok{1}\FloatTok{+0.02}\NormalTok{)}\OperatorTok{^}\NormalTok{x1,}
         \DataTypeTok{npv_base =}\NormalTok{ damage_base}\OperatorTok{/}\StringTok{ }\NormalTok{(}\DecValTok{1}\FloatTok{+0.02}\NormalTok{)}\OperatorTok{^}\NormalTok{x1,}
         \DataTypeTok{sum_dam_1.5 =} \KeywordTok{sum}\NormalTok{(npv_}\FloatTok{1.5}\NormalTok{), }
         \DataTypeTok{sum_dam_base =} \KeywordTok{sum}\NormalTok{(npv_base),}
         \DataTypeTok{exp_value =}\NormalTok{ ((sum_dam_}\FloatTok{1.5}\OperatorTok{*}\FloatTok{0.5}\NormalTok{)}\OperatorTok{+}\StringTok{ }\NormalTok{(sum_dam_base}\OperatorTok{*}\FloatTok{0.5}\NormalTok{)))}
\end{Highlighting}
\end{Shaded}

\begin{Shaded}
\begin{Highlighting}[]
\NormalTok{warming5B <-}\StringTok{ }\NormalTok{warming }\OperatorTok\StringTok{ }
\StringTok{  }\KeywordTok{select}\NormalTok{(x1, year, warming_baseline) }\OperatorTok\StringTok{ }
\StringTok{  }\KeywordTok{mutate}\NormalTok{(}\DataTypeTok{warming =} \KeywordTok{case_when}\NormalTok{(}
\NormalTok{    year }\OperatorTok{<}\StringTok{ }\DecValTok{2050} \OperatorTok{~}\StringTok{ }\NormalTok{warming_baseline,}
\NormalTok{    year }\OperatorTok{>=}\StringTok{ }\DecValTok{2050} \OperatorTok{~}\StringTok{ }\FloatTok{1.29}\NormalTok{),}
    \DataTypeTok{dam_b =}\NormalTok{ (warming}\OperatorTok{*}\DecValTok{3019000000000}\NormalTok{) }\OperatorTok{+}\StringTok{ }\NormalTok{((warming)}\OperatorTok{^}\DecValTok{2}\OperatorTok{*}\DecValTok{19590000000000}\NormalTok{),}
    \DataTypeTok{dam_b_sc =}\NormalTok{ dam_b}\OperatorTok{/}\StringTok{ }\NormalTok{(}\DecValTok{1}\FloatTok{+0.02}\NormalTok{)}\OperatorTok{^}\NormalTok{x1)}

\NormalTok{warming_5b_sum <-}\StringTok{ }\NormalTok{warming5B }\OperatorTok\StringTok{  }
\StringTok{ }\KeywordTok{mutate}\NormalTok{(}\DataTypeTok{sum_b_sc =} \KeywordTok{sum}\NormalTok{(dam_b_sc))}
\end{Highlighting}
\end{Shaded}

\begin{Shaded}
\begin{Highlighting}[]
\CommentTok{# Cost of Policy A = Cost of Policy B + x }
\CommentTok{# Need to find how much x could be for it still to make sense to use Policy B}

\CommentTok{# Maximum of x for Policy B to still make sense}
\NormalTok{exp_value_calc_a <-}\StringTok{ }\NormalTok{sum_prob_dam }\OperatorTok\StringTok{ }
\StringTok{  }\KeywordTok{select}\NormalTok{(exp_value)}

\NormalTok{exp_value_calc_b <-}\StringTok{ }\NormalTok{warming_5b_sum }\OperatorTok\StringTok{ }
\StringTok{  }\KeywordTok{select}\NormalTok{(sum_b_sc)}

\NormalTok{max_price_x =}\StringTok{ }\NormalTok{(exp_value_calc_a }\OperatorTok{-}\StringTok{ }\NormalTok{exp_value_calc_b)}
 
\CommentTok{#max_price_x}
\end{Highlighting}
\end{Shaded}

\begin{enumerate}
\def\labelenumi{\Alph{enumi})}
\item
  The expected present value of damages up to 2100 under Policy A is
  \$2.931942e+15
\item
  The expected present value of damages up to 2100 under Policy B is
  \$9.38138e+14
\item
  If society is risk neutral, Policy B would make economic sense as long
  as it is less than or equal to \$1.979412e+15 over the 80 year time
  frame. If the cost of implementing Policy B is above \$1.979412e+15,
  then Policy A would make more economic sense.
\item
  If society is risk averse, the maximum price of x could increase
  because the utility of guaranteed temperature reduction in Policy B is
  more than the utility of Policy A which involves risk.
\end{enumerate}

If society is instead risk averse, society is willing to pay more for
Policy B (up to a certain point) as it eliminates any risk that is
present in the business as usual Policy A. The certain point to which a
risk averse society would switch to Policy A would be when the price of
x would \_\_ the utility of policy of A is greater than the utility of
Policy B.

If A and B + x are equal the society will pick Policy B as the utility.

\end{document}
